%\documentclass[11pt,professionalfonts,hyperref={pdftex,pdfpagemode=none,pdfstartview=FitH}]{beamer}
%\usepackage{times}
\documentclass[11pt,professionalfonts]{beamer}
\usefonttheme{serif}
\usepackage{presentation_packages}
\usepackage[version=3]{mhchem}
\DeclareSIUnit\year{yr}
\newcommand{\hilight}[1]{\colorbox{green}{#1}}

\definecolor{mygray}{gray}{0.9}
\definecolor{RoyalBlue}{rgb}{0.25,0.41,0.88}
\def\Emph{\textcolor{RoyalBlue}}

\definecolor{tmp}{rgb}{0.804,0.941,1.0}
\setbeamercolor{numerical}{fg=black,bg=tmp}
\setbeamercolor{exact}{fg=black,bg=red}

\mode<presentation> 
{
  \usetheme{Warsaw}
  \usefonttheme{serif}
  \setbeamercovered{transparent}
}

\setbeamertemplate{footline}%{split theme}
{%
  \leavevmode%
  \hbox{\begin{beamercolorbox}[wd=.5\paperwidth,ht=2.5ex,dp=1.125ex,leftskip=.3cm,rightskip=.3cm plus1fill]{author in head/foot}%
    \usebeamerfont{author in head/foot}\insertshorttitle
  \end{beamercolorbox}%
  \begin{beamercolorbox}[wd=.5\paperwidth,ht=2.5ex,dp=1.125ex,leftskip=.3cm,rightskip=.3cm]{title in head/foot}
%    \usebeamerfont{title in head/foot}\mypaper\hfill \insertframenumber/\inserttotalframenumber
    \usebeamerfont{title in head/foot}\hfill \insertframenumber/\inserttotalframenumber
  \end{beamercolorbox}}%
  \vskip0pt%
} \setbeamercolor{box}{fg=black,bg=yellow}


\title[Autonomous Landing]{\large\textbf Geometric Control for Autonomous Landing on Asteroid Itokawa using Visual Imagery  }

\author{\vspace*{-0.3cm}}
   
\institute{
	\footnotesize
	{\normalsize\bf{Shankar Kulumani, Kuya Takami \\ Taeyoung Lee}}\\
	\vspace*{0.2cm}
  	\textbf{Department of Mechanical \& Aerospace Engineering}\\ \vspace*{0.5cm}
 	\begin{figure} %figure%
       	\includegraphics[width=0.75\textwidth]{gw_txh_2cs_pos}
  	\end{figure}
}
\date{\today}

\begin{document}
%=======================================================%

\setcounter{framenumber}{-1}
\begin{frame} %-----------------------------%
  \titlepage
\end{frame}   %-----------------------------%

\section*{}
\subsection*{Introduction}  
\begin{frame}{Asteroid Missions}
\begin{itemize}
    \item Science - insight into the early formation of the solar system
    \item Mining - vast quantities of useful materials
    \item Impact - high risk from hazardous near-Earth asteroids
\end{itemize}    

\begin{center}
    \includegraphics[height=0.35\textheight]{figures/near_mos_20001203_full.jpg}
    \hfill
    \includegraphics[height=0.35\textheight]{figures/Itokawa8_hayabusa_1210.jpg}
\end{center}
\end{frame}

\begin{frame}{Asteroid Mining}
    \begin{itemize}
      \item Useful materials can be extracted from asteroids to support:
      \begin{itemize}
          \item Propulsion, construction, life support, agriculture, and precious/strategic metals
      \end{itemize}
      \item Commercialization of near-Earth asteroids is feasible
    \end{itemize}

\pause

\begin{center}
\small
    \begin{tabular}{|l|r|r|}
        \hline 
        Element & Price (\SI{}[\$\,]{\per\kilo\gram}) & Sales (\SI{}[\$\,]{M\per\year}) \\
        \hline \hline 
        Phosphorous (P) & \num{0.08}  & \num{2167} \\
        Gallium (Ga) & \num{300.00}  & \num{1544} \\
        Germanium (Ge) & \num{745.00} & \num{6145} \\
        \hline \hline 
        Platinum (Pt) & \num{12394.00} & \num{1705} \\
        Gold (Au) & \num{12346.00} & \num{49} \\
        Osmium (Os) & \num{12860.00} & \num{307} \\
        \hline
    \end{tabular}
\end{center}

\end{frame}

\section*{}
\subsection*{Motivation}  

\section*{Mathematical Background}
\subsection*{Coupled Dynamics}
\begin{frame}{Gravitational Modeling} %-----------------------------%

\begin{itemize}
  \item Asteroids are extended bodies not point masses
  \pause
  \item Spherical Harmonic - only valid outside of circumscribing sphere
    {
    \small
    \begin{align*}
      U(\vecbf{r} ) = \frac{\mu}{r} \sum_{n=0}^\infty \sum_{m=0}^\infty \parenth{\frac{R}{r}}^nP_{n,m}(\sin \phi) \braces{ C_{nm} \cos(m \lambda) + S_{nm} \sin(m \lambda)} 
    \end{align*}
    }
    \pause
  \item Infinite series is an approximation and adds complexity
    \begin{itemize}
        \item Model switching at circumscribing sphere
        \item Coefficient matching used to ensure continuity 
    \end{itemize}
\end{itemize}

\note[itemize]{
  \item Models require detailed data from orbit about asteroid (OD process determines gravity field)
  \item Simplified models (triaxial ellipsoid allows analytical insight)
  \item Previous work fails to consider coupled dyanmics
  }

\end{frame}   %-----------------------------%

\section*{}
\subsection*{System Model}

\begin{frame}{Polyhedron Gravitation Model}

\begin{itemize}
    \item Potential is a function of only the shape model
    \item Globally valid, closed-form expression of potential
    \item Exact potential assumes a constant density 
    \item Accuracy solely dependent on shape model
\end{itemize}
\only<2>{
\begin{align*}\label{eq:potential}
    U(\vecbf{r}) &= \frac{1}{2} G \sigma \sum_{e \in \text{edges}} \vecbf{r}_e \cdot \vecbf{E}_e \cdot \vecbf{r}_e \cdot L_e - \frac{1}{2}G \sigma \sum_{f \in \text{faces}} \vecbf{r}_f \cdot \vecbf{F}_f \cdot \vecbf{r}_f \cdot \omega_f 
\end{align*}
}

\end{frame}
\subsection*{Polyhedron Potential Model}

\section*{}
\subsection*{Geometric Control}

\section*{}
\subsection*{Blender}

\subsection*{Structure from Motion}

\section*{}
\subsection*{Numerical Simulation}

\section*{}
\subsection*{Conclusions}

\end{document}

